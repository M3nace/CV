%%%%%%%%%%%%%%%%%%%%%%%%%%%%%%%%%%%%%%%%%
% Twenty Seconds Resume/CV
% LaTeX Template
% Version 1.0 (14/7/16)
%
% Original author:
% Carmine Spagnuolo (cspagnuolo@unisa.it) with major modifications by
% Vel (vel@LaTeXTemplates.com) and Harsh (harsh.gadgil@gmail.com)
%
% License:
% The MIT License (see included LICENSE file)
%
%%%%%%%%%%%%%%%%%%%%%%%%%%%%%%%%%%%%%%%%%

%----------------------------------------------------------------------------------------
%	PACKAGES AND OTHER DOCUMENT CONFIGURATIONS
%----------------------------------------------------------------------------------------

\documentclass[letterpaper]{twentysecondcv} % a4paper for A4

% Command for printing skill overview bubbles
\newcommand\skills{
  \smartdiagram[bubble diagram]{
    \textbf{Développeur}\\\textbf{cybersécurité},
    \textbf{Gestion}\\\textbf{de projet},
    \textbf{~~~~~~SIEM~~~~~~~},
    \textbf{~~~~~IDMEF~~~~~},
    \textbf{Méthodes}\\\textbf{agiles},
    \textbf{Avant}\\\textbf{vente}
  }
}

% Programming skill bars
\programming{{C $\textbullet$ C++ $\textbullet$ \large \LaTeX / 2.5}, {SQL $\textbullet$ CSS $\textbullet$ HTML / 4}, {Python $\textbullet$ JavaScript / 5}}

% Projects text
\education{
\textbf{Expert en technologies de l'information} \\
EPITECH Paris \\
2008 - 2014 | Paris, France

\textbf{Information technology \& Computer science} \\
Ahlia University \\
2012 - 2013 | Manama, Bahrein
}

%----------------------------------------------------------------------------------------
%	 PERSONAL INFORMATION
%----------------------------------------------------------------------------------------
% If you don't need one or more of the below, just remove the content leaving the command, e.g. \cvnumberphone{}

\cvname{Camille GARDET} % Your name
\cvjobtitle{Ingénieur développement} % Job
% title/career

\cvlinkedin{/in/cgardet}
\cvgithub{M3nace}
\cvnumberphone{06 70 15 71 59} % Phone number
%\cvsite{} % Personal website
\cvmail{gardet.ca@gmail.com} % Email address

%----------------------------------------------------------------------------------------

\begin{document}

\makeprofile % Print the sidebar

%----------------------------------------------------------------------------------------
%	 EXPERIENCE
%----------------------------------------------------------------------------------------

\section{Expériences}

\begin{twenty} % Environment for a list with descriptions
  \twentyitem
      {Mai 2014}
      {Aujourd'hui}
      {Ingénieur d'études intégrateur}
      {CS}
      {}
      {
        \bigskip
        Participation active dans le développement du produit \href{https://www.prelude-siem.com/}{Prelude SIEM} et dans les projets adjacents :
        \\
        \begin{itemize}
        \item Recrutement et suivi des stagiaires
        \item Participation à l'avant-vente du produit et réponse à appel d'offre
        \item Participation aux salons du secteur
        \item R\&D et innovation sur des fonctionnalités inédites du SIEM
        \item Accompagnement et support client au travers de POC
        \item Veille technique autour des éléments composant le SIEM comme les règles de détection, règles de corrélation, éléments web, etc.
        \end{itemize}
        \ \\
        Promotion du format IDMEF, à travers le projet SECEF :
        \\
        \begin{itemize}
        \item Rédaction d'études et de leurs résultats sur le format interopérable IDMEF et de sa nouvelle version
        \item Distribution au format ouvert des outils de manipulation du format
        \item Implémentation d'exemple d'utilisation et re-distribution à la communauté open source sur (\href{https://github.com/IDMEF-IODEF}{https://github.com/IDMEF-IODEF})
        \end{itemize}
      }
      \\

  \twentyitem
      {Sept.}
      {Déc. 2013}
      {Professeur assistant}
      {ETNA}
      {}
      {
        \bigskip
        \begin{itemize}
        \item Encadrement des cours et des TP
        \item Mise en place et application des tests d’évaluations
        \end{itemize}
      }
      \\

 \twentyitem
     {Mai}
     {Août 2012}
     {Stage en développement logiciel}
     {CS}
     {}
     {
       \bigskip
       \begin{itemize}
       \item Conception et réalisation d’un outil de test pour la norme ED137 (nouveau protocole de communication)
       \item Test et validation du système existant
       \end{itemize}
     }
     \\

 \twentyitem
     {Sept. 2011}
     {Avril 2012}
     {Stage en administration réseau}
     {Thales-Raytheon Systems}
     {}
     {
       \bigskip
       \begin{itemize}
       \item Création d’un réseau IP supportant le transport de paquets X.25 et BSC
       \item Intégration dans le système existant
       \item Commande et renouvellement du matériel à des fins de mise à jour
       \end{itemize}
     }
     \\

 \twentyitem
     {Août}
     {Déc. 2009}
     {Stage en développement web}
     {Thales-Raytheon Systems}
     {}
     {
       \bigskip
       \begin{itemize}
       \item Création d’un panneau de maintenance pour la supervision de plateformes et systèmes complexes
       \item Fournir une nouvelle technologie au système existant pour accroître les performances et les possibilités d'évolutions
       \end{itemize}
     }

      %\twentyitem{<dates>}{<title>}{<location>}{<description>}
\end{twenty}

\end{document}
